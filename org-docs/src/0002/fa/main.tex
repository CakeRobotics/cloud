\documentclass{cake-classes/short-report-fa}
\usepackage{booktabs}
\begin{document}
\درج‌عنوان‌سند

\قسمت{مقدمه}
در این متن، معماری پیشنهادی برای مجموعه مستندات داخلی پروژهٔ کیک روباتیک معرفی شده است.

\قسمت{بیان مساله}
هدف این است که تولید مستندات داخلی پروژه و استفاده از آن‌ها آسان باشد.
این هدف به صورت زیر تحقق می‌یابد:
\شروع{فقرات}
\فقره تولید مستندات باید آسان باشد.
\فقره دسترسی مستندات باید آسان باشد.
\فقره جستجو در مستندات باید آسان باشد.
\پایان{فقرات}

قیدهای اضافی زیر نیز باید رعایت شوند:
\شروع{فقرات}
\فقره دسترسی به مستندات داخلی باید قابل کنترل و محدودسازی باشد.
\پایان{فقرات}

\قسمت{طراحی}
در معماری پیشنهادی، مستندات داخلی تیم به سه دستهٔ زیر تقسیم شده است:
\شروع{شمارش}
\فقره \مل{API} میکروسرویس‌ها:
مستنداتی که در سطح سرویس‌ها هستند در این دسته قرار می‌گیرند.
این دسته مستندات راه اصلی ارتباط اعضای تیم هستند.
به ازای هر میکروسرویسی که در تیم توسعه داده می‌شود، یک مجموعه سند جدید در این دسته اضافه می‌شود و تمام موارد پیاده‌سازی شده در سرویس مربوطه در آن دسته مستند می‌شود.
این مستندات پیوسته توسط توسعه‌دهندگان سرویس‌ها به روز می‌شوند.
\فقره مستندات سازمان:
مستنداتی که بالاتر از سطح سرویس‌ها هستند در این دسته قرار می‌گیرند.
این مستندات شامل گزارش‌ها، مستندات مربوط به کسب و کار، مستندات معماری سطح بالا، سندهای توجیهی و مواردی از این دست می‌شوند.
برای نمونه، گزارش پیش رو متعلق به این گروه است.
این سندها پس از رسیدن به یک نسخهٔ پایدار، دیگر تغییر کلی نخواهند کرد و در صورت نیاز با برچسب‌هایی نظیر منسوخ شده یا الحاق یافته، به روز می‌شوند.
\فقره مستندات زنده:
مستنداتی که بالاتر از سطح سرویس‌ها هستند و به شدت تغییر می‌کنند در این دسته قرار می‌گیرند.
این مستندات عملا مانند یک تختهٔ دیواری هستند و شامل مواردی نظیر لیست وظایف در حال انجام، نمودار پیشروی، وضعیت زندهٔ سرویس‌ها و... می‌شوند.
\پایان{شمارش}

الزامات هر یک از این دسته‌ها منحصر به فرد است و در جدول \رجوع{الزامات} آمده است. بر اساس این الزامات، طراحی کلی هر دسته تحقق یافت که در جدول \رجوع{طراحی} آمده است.

\begin{table}[hp]
\شرح{الزامات هر دسته از مستندات}
    \vspace{0.5cm}
	\centering
	\begin{tabular}{rp{10cm}}
\toprule
دسته & الزامات \\
\midrule
\مل{API} میکروسرویس‌ها & - انتشار و بروزرسانی نباید مستلزم عملیات دستی فردی به جز نویسنده باشد. \\
& - در صورت نیاز باید قابلیت تولید مستندات اتوماتیک بر اساس کامنت‌های کد وجود داشته باشد. \\
&  - باید قابلیت مدیریت نسخه‌های موازی را داشته باشد. \\
&  - باید قابلیت لینک‌دهی در سطح مستندات سرویس را داشته باشد. \\
&  - باید در سطح کلمه قابل جستجو باشد. \\
\hline
مستندات سازمان &  - باید توانایی تولید سندهای پیچیده و ناهمگن را داشته باشد. \\
&  - باید یک نمایهٔ مرکزی داشته باشد. \\
&  - باید سطح دسترسی به هر سند به طور مستقل قابل تعریف باشد. \\
&  - بهتر است قابل چاپ باشد. \\
\hline
مستندات زنده &  - باید قابلیت به روز رسانی سریع را داشته باشد. \\
&  - باید سطح دسترسی برای انتشار و همچنین خواندن هر سند به طور مستقل قابل تعریف باشد. \\
\bottomrule
	\end{tabular}
\برچسب{الزامات}
\end{table}

\begin{table}[hp]
	\شرح{طراحی هر دسته از مستندات}
	\vspace{0.5cm}
	\centering
	\begin{tabular}{rp{6cm}rr}
		\toprule
		دسته & شرح & متمرکز/غیرمتمرکز & زبان‌های رسمی \\
		\midrule
\مل{API} میکروسرویس‌ها & هر فرد یا گروه آزاد است که فناوری مورد علاقهٔ خود را برای مستندسازی \مل{API} سرویسی که توسعه می‌دهد انتخاب کند. تنها قید این است که مستندات مورد نظر باید با \مل{git} نسخه‌گذاری شود. یک نمایهٔ مرکزی برای دسترسی به صفحهٔ اصلی مستندات هر سرویس وجود خواهد داشت. همچنین، پلتفرم پیشنهادی برای نوشتن این مستندات، \مل{sphinx} است. & غیرمتمرکز & انگلیسی \\
		\hline
مستندات سازمان & در قالب فایل‌های لاتک در \مل{git} نگهداری می‌شوند و یک سرویس اضافی برای تولید نمایه، تولید \مل{pdf}ها و همچنین مدیریت دسترسی توسعه داده می‌شود. & متمرکز & فارسی، انگلیسی \\
		\hline
مستندات زنده & مدیریت تسک‌ها در \مل{Trello} انجام می‌شود. سایر اسناد زنده در یک حساب \مل{Google Docs} مرکزی قرار می‌گیرد. & متمرکز & انگلیسی \\
		\bottomrule
	\end{tabular}
	\برچسب{طراحی}
\end{table}

\قسمت{موارد توجیهی}
استفاده از لاتک برای نوشتن مستندات سازمان باعث می‌شود مستندات پیچیده که بعضی از آن‌ها ممکن است گزارش‌های کوتاه دو صفحه‌ای باشند و برخی از آن‌ها ممکن است بالای صد صفحه باشند، در یک نمایهٔ متمرکز قابل ثبت و دسترسی باشند. در صورت انتخاب فناوری‌های بر پایهٔ‌ صفحات وب، کیفیت خروجی پایین‌تری به دست می‌آمد و مطالعهٔ سندهای طولانی دشوار می‌شد. قابل چاپ بودن لاتک نیز نکتهٔ مثبتی است. در مقایسه با \مل{Word} نیز، لاتک جدایش محتوا از قالب را بهتر فراهم می‌کند، قابلیت گسترش بیشتری دارد، و جستجو در آن آسان است.

برای میکروسرویس‌ها انتخاب به ازای هر سرویس به توسعه دهندهٔ مربوطه واگذار شده است. دلیل این امر این است که نگهداری تمام مستندات در یک سامانهٔ مرکزی با گسترش پروژه دشوار می‌شود. همچنین، مستقل بودن کامل مستندات باعث ایزوله شدن سرویس‌ها و استقلال آن‌ها می‌شود که در معماری میکروسرویس بسیار مطلوب است. این امر که این پلتفرم‌ها اغلب از نوع \مل{Markdown} هستند باعث می‌شود به آسانی قابل جستجو هستند. در صورت تمایل به عمومی‌سازی یک سرویس، کار اضافی لازم نخواهد بود. در نهایت، غیرمتمرکز بودن پلتفرم‌های مستندسازی \مل{API} باعث می‌شود افراد بتوانند بر اساس فناوری‌های دیگری که در سرویس خود استفاده می‌کنند پلتفرم مستندسازی را انتخاب کنند. برای مثال، عموما پلتفرم \مل{sphinx} برای پایتون مناسب‌تر است در حالی که پلتفرم مناسب برای \مل{C++}، \مل{doxygen} است.

در مورد مستندات زنده با توجه به لزوم آسانی به روز رسانی سریع، \مل{Google Docs} بهترین انتخاب است. همچنین قابلیت کنترل دسترسی به طور کامل در \مل{Google Docs} وجود دارد و کار اضافی لازم نخواهد بود. با این حال، برای کاربرد خاص مدیریت \مل{Scrum}، پلتفرم \مل{Trello} پیشنهاد شد تا از بهینه‌سازی‌های آن برای این کار استفاده شود.

\end{document}
